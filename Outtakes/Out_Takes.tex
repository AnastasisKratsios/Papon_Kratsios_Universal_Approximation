\begin{theorem}[{Condition~\eqref{eq_local_curvature_condition} Cannot be Omitted}]\label{thrm_negative_motiation}
Let $\xxx=\rrflex{m+1}$, $\yyy=S^m$, $\kkk=S^n\subset \rrflex{m+1}$, and fix any continuous activation function.  Then, for every $k \in \nn$, $\NN[\rrflex{m+1},S^m;k]$ is a well-defined subset of $C(\kkk,S^m)$, but, there exists some $\epsilon>0$ for which
$$
\underset{
\hat{f} \in \NN[\rrflex{m+1},S^m;k]
}{\inf}\,
\sup_{x \in \kkk}\,
d_{S^n}\left(
\hat{f}(x),1_{S^n}(x)
\right)\geq \epsilon;
$$
where $1_{S^n}$ denotes the identity map on $S^n$.  
\end{theorem}

\begin{proof}[{Proof of Theorem~\ref{thrm_negative_motiation}}]
We first remark that since the activation function $\sigma$ is continuous then, for every $k\in \nn$, every $g \in \NN[m+1,m,k]$ is a continuous function from $\rrflex{m+1}$ to $\rrm$.  
Next, we remark that since $S^m$ is a metric space (with Riemannian distance function $d_{S^m}$) then \citep[Theorem 46.8]{munkres2014topology} implies that the uniform topology on $C(S^m,S^m)$ and the compact-open topology thereon, generated by the sub-basic open sets $\{U_{K,O}:\,K\subseteq S^m\mbox{ compact and } O\subseteq S^m \mbox{ open}\}$ where
$$
U_{K,O}\triangleq 
\left\{
f\in C(S^m,S^m):\,
f(K)\subseteq O
\right\},
$$
coincide.  We therefore consider $C(S^m,S^m)$ with the latter of the two.   

Since $\xxx=S^m=\yyy$ then by \citep[Example 0.3]{HatcherAlgebraicTopology} both $\xxx$ and $\yyy$ are CW-complexes (see \citep[page 5]{HatcherAlgebraicTopology}).  Moreover, since $\xxx$ is a Riemannian manifold then it is a metric space, with (Riemannian) distance function $d_{\xxx}$.  Furthermore, since $S^m$ is closed and bounded in $\rrflex{m+1}$ then by the Heine-Borel Theorem it is compact.  Hence, \citep[Corollary 2]{MilnorHomotopyTypeCWComplexMappingSpace1959} applies and therefore $C(\xxx,\yyy)$ is a CW-complex whose path components correspond the homotopy classes in $C(\xxx,\yyy)$.  By \citep[Proposition A.4]{HatcherAlgebraicTopology} every CW-complex is locally-contractible and therefore (see \citep[page 522]{HatcherAlgebraicTopology}) it is locally path-connected.  Hence, by \citep[Theorem 25.5]{munkres2014topology} every path component of $C(\xxx,\yyy)$ is a connected component and therefore every path component of $C(\xxx,\yyy)$ is closed.  Hence, every homotopy class in $C(\xxx,\yyy)$ is closed.  

Therefore, it is enough to demonstrate that that, for every $y \in S^m$, $k\in \nn$, and every $g \in \NN[m+1,m,k]$, the function $f=\operatorname{Exp}_{S^m,y}^{-1}\circ g$ is not homotopic to the identity function $1_{S^m}$.  We accomplish this by showing that $f$ is homotopic to a constant function while the identity is not.  Therefore, since homotopy equivalence is a transitive relation, then any such $f$ and $1_{S^m}$ cannot be homotopic.  

Let us begin by determining the homotopy class of $1_{S^m}$.  Recall that a space is, by definition, contractible if and only if its identity function is homotopic to a constant function.  However, by the first corollary to Hopf's Theorem \citep[page 125]{FuchsFomenkoHomotopicalTopology2016Edition2} (there is no numbering) $S^m$ is not contractible and therefore $1_{S^m}$ does not lie in the same homotopy class as any constant function.  

Next, show that any non-Euclidean feed-forward network is contractible.  Indeed, let $y \in S^m$, $k \in \nn$, and let $g\in \NN[m+1,m,k]$.  Observe that for any $t \in [0,1]$ the function 
$$
g_t(x)\triangleq tg(x),
$$
is also a feed-forward network; as multiplication by the scalar $t$ simply corresponds to re-scaling last affine layer map defining $g$; thus, $g_t \in \NN[m+1,m,k]$.  Moreover, since multiplication is continuous in $\rrflex{m}$ and the composition of continuous functions is again continuous then the map
$$
\begin{aligned}
 G:[0,1]\times \rrflex{m+1}:    
.
\end{aligned}
$$
In particular its restriction to $[0,1]\times S^m$ is continuous and takes values in $S^m$.  Hence, $G$ defines a homotopy from $g$ to the constant function $g_0(x)\mapsto \operatorname{Exp}_{S^m,y}(0)$.  Therefore, for every $y \in S^m$, every $k \in \nn$, every continuous activation function $\sigma$, and every $g\in \NN[m+1,m,k]$, 
the function $\hat{f}:x\mapsto \operatorname{Exp}_{S^m,y}\circ g$ is homotopic to a constant function.  Therefore, it is not homotopic to $1_{S^m}$ and therefore the sets $\{1_{S^m}\}$ and $F\triangleq \left\{\operatorname{Exp}_{S^m,y}\circ g:\, k\in \nn,\, y\in S^m,\, g \in \NN[m+1,m,k]\right\}$ lie in different components of $C(S^m,S^m)$.  

Again applying \citep[Theorem 46.8]{munkres2014topology}, the thus must exist some $0<\epsilon$ bounding the uniform distance by these sets; i.e.:
$$
\epsilon\leq 
\underset{{\hat{f}\in F}}{\inf}\,
\underset{x \in S^m}{\sup}
\,
d_{S^m}\left(
f(x), \hat{f}(x)
\right) 
.
$$
\end{proof}